\documentclass{tccv_full}
\usepackage[english]{babel}
\usepackage{epsfig}
\usepackage{wrapfig}
% im_convert.exe "C:\Users\Ricardo\Google Drive\Documentos\CV\FullSizeRender.png" eps3:"C:\Users\Ricardo\Google Drive\Documentos\CV\FullSizeRender.eps"

\begin{document}

\part{Marcos Blandim Andrade}
\personal[github.com/marcosblandim]
{Bras\'ilia-DF, Brasil}
{+55 (61) 98109-9774}
{marcosblandim@gmail.com}
{linkedin.com/in/marcosblandim/}


\vspace{0.5cm}

\section{Sumário}

Neste resumo você pode achar as tecnologias e tópicos que domino e as atividades das quais participei, como o Laboratório UIOT da UnB, em que atuo como pesquisador.

\section{Experiências}
\begin{itemize}
	\item{\large Maio 2019 -- Presente:	\textbf{Pesquisador no Laboratório IoT da UnB}}
	\begin{itemize}
		\item \textsf{Desenvolviemnto de robô autônomo guiado por visão computacional}
		{\small
			\subitem - OpenCV
			\subitem - Python
			\subitem - RaspberryPI
		}
		\item \textsf{Criação de sistema de automação de ambientes usando sensores e IA}
	\end{itemize}
	\vspace{0.3cm}
	\item{\large Outubro 2019 -- Presente:	\textbf{Membro do ramo estudantil ComSoc da IEEE na UnB}}
	\begin{itemize}
	    \item \textsf{Desenvolvimento de sistema web utilizando django, html e css}
	\end{itemize}
	\vspace{0.3cm}
	\item{\large Agosto 2019 -- Deszembro 2019:	\textbf{Monitor da disciplina Algoritmos e Programação de Computadores na UnB}}
\end{itemize}

\section{Tecnologias}

\begin{factlist}

\item{Intermediário}
     {Python, Django, C, C++, Linux, Git, Rasperry Pi, OpenCV}

\item{Básico}
     {HTML, CSS, Javascript}



\end{factlist}

\section{Conhecimentos}

\begin{factlist}

\item{Intermediário}
     {Desenvolvimento Web, Estrutura de Dados, Programação Orientada a Objetos}

\item{Básico}
     {Banco de Dados, Internet of Things, Processamento de Imagens, Redes de computadores}

\end{factlist}

\section{Educação}

	{Março 2018 -- Presente}: Cursando o 5º semestre de  \textbf{Engenharia da Computação} na Universidade de Brasília

\section{Línguas}

\begin{factlist}
\item{Português}{Língua materna}
\item{Inglês}{Escreve, compreende e fala bem}
\item{Francês}{Escreve, compreende e fala pouco}
\end{factlist}

\end{document}
